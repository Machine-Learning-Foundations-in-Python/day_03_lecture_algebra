% no notes
%\documentclass{beamer}
% notes and slides
\documentclass[notes]{beamer}
% notes only
% \documentclass[notes=only]{beamer}
\usepackage{graphicx} % Allows including images
\usepackage{booktabs} % Allows the use of \toprule, \midrule and \bottomrule in tables
\usepackage{multirow}
\usepackage{multimedia}
\usepackage{tikz}
\usepackage{circuitikz}
\usepackage{url}
\usepackage{pgfplots}
\pgfplotsset{compat=newest}
\usepgfplotslibrary{groupplots,dateplot}
\usetikzlibrary{patterns,shapes.arrows}
\usepackage{standalone}
\usepackage{adjustbox}
\usepackage{lmodern}
\usepackage{pgfplots}
\usepackage{amsmath}
\usepackage{amsthm}
\usepackage{multimedia}
\usepackage{standalone}
\usepackage{csquotes}


\PassOptionsToPackage{american}{babel} % change this to your language(s), main language last
% Spanish languages need extra options in order to work with this template
% \PassOptionsToPackage{spanish,es-lcroman}{babel}
\usepackage{babel}

\PassOptionsToPackage{%
  backend=biber,bibencoding=utf8, %instead of bibtex
  %backend=bibtex8,bibencoding=ascii,%
  language=auto,%
  style=numeric-comp,%
  %style=authoryear-comp, % Author 1999, 2010
  %bibstyle=authoryear,dashed=false, % dashed: substitute rep. author with ---
  style=alphabetic,
  sorting=nyt, % name, year, title
  maxbibnames=10, % default: 3, et al.
  %backref=true,%
  %natbib=true % natbib compatibility mode (\citep and \citet still work)
}{biblatex}
\usepackage{biblatex}

\addbibresource{bib.bib}

\usetheme{metropolis}           % Use metropolis theme
\setbeamertemplate{caption}[default]
\title{Linear Algebra for Machine Learning in Python}
\date{\today}
\institute{High Performance Computing and Analytics Lab}
\author{Dr. Moritz Wolter}

\titlegraphic{\includegraphics[width=2.00cm]{UNI_Bonn_Logo_Standard_RZ.pdf}}
\begin{document}
    \maketitle

    \begin{frame}
    \frametitle{Overview} 
    \tableofcontents
    \end{frame}

    \section{Introcution}
    \begin{frame}{Motvating linear algebra}
      TODO
    \end{frame}

    \begin{frame}{Matrices}
      $m,n$ is a matrix $\mathbf{A}$
      \begin{align}
        \mathbf{A} = \begin{pmatrix}
          a_{11} & a_{12} & \dots & a_{1n} \\
          a_{21} & a_{22} & \dots & a_{2n} \\
          \vdots & \vdots &       & \vdots \\
          a_{m1} & a_{m2} & \dots & a_{mn}
        \end{pmatrix}
      , a_{ij} \in \mathbb{R}.
      \end{align}
     \end{frame}


    \section{Essential operations}
    \begin{frame}{Addition}
      TODO
    \end{frame}

    \begin{frame}{Multiplication}
      TODO
    \end{frame}

    \begin{frame}{Motivation of the determinant}
      TODO
    \end{frame}

    \begin{frame}{Computing determinants in two or three dimensions}
      The two dimensional case:
      \begin{align}
        \begin{vmatrix}
          a_{11} & a_{12} \\
          a_{21} & a_{22} \\
        \end{vmatrix}
        = a_{11} \cdot a_{22} - a_{12} \cdot a_{21} \\
      \end{align}
      Computing the determinant of a three dimensional matrix.
      \begin{align}
        \begin{vmatrix}
          a_{11} & a_{12} & a_{13}  \\
          a_{21} & a_{22} & a_{23}  \\
          a_{31} & a_{32} & a_{33}
        \end{vmatrix}
        = a_{11} \cdot
         \begin{vmatrix}
          a_{21} & a_{23}   \\
          a_{32} & a_{33}   \\
         \end{vmatrix}  
         -
         a_{21} \cdot
         \begin{vmatrix}
          a_{12} & a_{13}   \\
          a_{32} & a_{33}   \\
         \end{vmatrix}  
        +
         a_{31} \cdot
         \begin{vmatrix}
          a_{12} & a_{13}   \\
          a_{22} & a_{23}   \\
         \end{vmatrix}  
      \end{align}

      \note{Draw the sign pattern on the board:
      \begin{align}
        \begin{vmatrix}
          + & - & + & \dots \\
          - & + & - &  \dots \\
          + & - & + &  \dots \\
          \vdots & \vdots & \vdots & \ddots \\
        \end{vmatrix}        
      \end{align}
      The determinant can be expandend along any column as long as the sign pattern is respected.
      }
    \end{frame}

    \begin{frame}{Determinants in n-dimensions}
      \begin{align*}
      \begin{vmatrix}
        a_{11} & a_{21} & \dots & a_{1n} \\
        a_{21} & a_{22} & \dots & a_{2n} \\
        \vdots & \vdots &       & \vdots \\
        a_{m1} & a_{m2} & \dots & a_{mn}
      \end{vmatrix}
    = a_{11} \begin{vmatrix}
      a_{22} & \dots & a_{2n} \\
      \vdots &        & \vdots \\
      a_{m2} & \dots & a_{mn}
    \end{vmatrix}
    + a_{21} 
    \begin{vmatrix}
      a_{21} & \dots & a_{2n} \\
      \vdots &        & \vdots \\
      a_{m2} & \dots & a_{mn}
    \end{vmatrix}
    \\
    - a_{m1}
    \begin{vmatrix} 
      a_{11} & \dots & a_{1n} \\
      a_{21} & \dots & a_{2n} \\
      \vdots &        & \vdots \\
    \end{vmatrix}
    \end{align*}
    \end{frame}

  \begin{frame}{The Transpose}
      TODO
  \end{frame}

  \begin{frame}{The Inverse}
    TODO
  \end{frame}

  \section{Linear curve fitting}
    \begin{frame}{What is the best line connecting measurements?}
    
    \end{frame}

    \begin{frame}{The Pseudoinverse}
        \begin{align}
          \mathbf{A}^{\dagger} = (\mathbf{A}^T\mathbf{A})^{-1}\mathbf{A}^T
        \end{align}
    \note{
      Sometimes solving $\mathbf{A}\mathbf{x} + \mathbf{b} = 0$ is implossible.
      One the board, derive:
      \begin{align}
        \min_x \dfrac{1}{2}|\mathbf{A}\mathbf{x} - \mathbf{b}|^2 \\
      \end{align}
      At the optimum we expect,
      \begin{align}
        0 &= \nabla_x \dfrac{1}{2}|\mathbf{A}\mathbf{x} - \mathbf{b}|^2 \\
          &= \nabla_x \dfrac{1}{2}(\mathbf{A}\mathbf{x} - \mathbf{b})^T(\mathbf{A}\mathbf{x} - \mathbf{b}) \\
          &= (\mathbf{A}\mathbf{x} - \mathbf{b})\mathbf{A}^T \\
          &= \mathbf{A}^T\mathbf{A}\mathbf{x} - \mathbf{A}^T\mathbf{b} \\
          \mathbf{A}^T\mathbf{b} &= \mathbf{A}^T\mathbf{A}\mathbf{x} \\
          (\mathbf{A}^T\mathbf{A})^{-1}\mathbf{A}^T\mathbf{b} &= \mathbf{x}  
      \end{align}
    }
    \end{frame}

  \section{System dynamics, and dimensions}
    \begin{frame}{Eigenvalues}
      TODO
  \end{frame}


\end{document}
